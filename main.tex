\documentclass[a4paper,12pt,oneside,onecolumn,final,fleqn]{repUERJ}

\usepackage[english,brazil]{babel}  % adequação para o português Brasil
\usepackage[utf8]{inputenc} % Determina a codificação utilizada
                            % (conversão automática dos acentos)
\usepackage{makeidx}        % Cria o índice
\usepackage{hyperref}       % Controla a formação do índice
\usepackage{indentfirst}    % Endenta o primeiro paragrafo de
                            % cada seção.
\usepackage{graphicx}       % Inclusão de gráficos
\usepackage{subfig}
\usepackage{amsmath}        % pacote matemático
\usepackage{bm}             % pacote de fontes matematicas
\usepackage{lscape}         % pacote para colocar páginas em landscape
\usepackage{array}

\usepackage{float}
\usepackage{geometry}
\usepackage{pgfgantt}
\usepackage{setspace}

% ---
% Pacote auxiliar para as normas da UERJ
% ---
\usepackage[frame=no,font=default]{repUERJformat}
\usepackage[line=yes]{repUERJpseudocode} 
% ---
% Pacotes de citacoes
% ---
\usepackage[alf,abnt-repeated-author-omit=no]{abntex2cite}
% ---
\selectlanguage{brazil} 


% ********************************************************************
% ********************************************************************
% Área Reservada para incluir os novos comandos
% ********************************************************************
% ********************************************************************

% Comandos de comentários
\definecolor{green}{rgb}{0.1, 0.4, 0.1}
\newcommand\red[1]{{\color{red}#1}}
\newcommand\sred[1]{{\color{red}\uline{#1}}}
\newcommand\blue[1]{{\color{blue}#1}}
\newcommand\sblue[1]{{\color{blue}\uline{#1}}}
\newcommand\green[1]{{\color{green}#1}}
\newcommand\sgreen[1]{{\color{green}\uline{#1}}}
\newcommand\orange[1]{{\color{orange}#1}}
\newcommand\sorange[1]{{\color{orange}\uline{#1}}}
%o comando \sout funciona para riscar o texto

% ********************************************************************
% ********************************************************************
% Informações de autoria e institucionais
% ********************************************************************
% ********************************************************************

%---------------------------------------------------------------------
% Imagens pretextuais (precisam estar no mesmo diretório deste arquivo .tex)
%---------------------------------------------------------------------
 
\logo{logo_uerj_cinza.png}
\marcadagua{marcadagua_uerj_cinza.png}{1}{160}{255}

%---------------------------------------------------------------------
% Informações da instituição
%---------------------------------------------------------------------
\instituicao{Universidade do Estado do Rio de Janeiro}  %Universidade
            {Centro de Tecnologia e Ciências}  %Centro
            {Faculdade de Engenharia} %Unidade
            {} %Departamento (DEIXAR EM BRANCO)

%---------------------------------------------------------------------
% Informações da autoria do documento
%---------------------------------------------------------------------

\autor{Petro}
      {Liporace Cardoso de Souza}
      {P. L. C. de S.} % iniciais do nome

\titulo{Escalabilidade em Aplicações Globais em Tempo Real: Um Estudo de Caso com o Projeto Codeboard UERJ} %Título do trabalho acadêmico em português
\title{Scalability in Global Real-Time Applications: A Case Study with the Codeboard UERJ Project} %Título do trabalho acadêmico em inglês



\orientador{Prof.ª} %cargo, ex.: Prof., Profa., Eng. , etc...
		   {Rafaela}{Correia Brum} %Nome sobrenome com a titulação ao final. Exemplo: D.Sc., Ph.D., M.Sc., B.Sc., etc.
           {Universidade Federal Fluminense - UFF} % Instituição e Departamento ou PPG


%Opcional, Comente as linhas de coorientador caso não tenha
\coorientador{Prof.ª}  %cargo, ex.: Prof., Profa., Eng. , etc...
           {Cristiana}{Barbosa Bentes}  %Nome sobrenome com a titulação ao final. Exemplo: D.Sc., Ph.D., M.Sc., B.Sc., etc.
           {Departamento de Engenharia de Sistemas e Computação - UERJ} % Instituição e Departamento ou PPG

%---------------------------------------------------------------------
% Grau pretendido (Doutor, Mestre, Bacharel, Licenciado) e Curso
%---------------------------------------------------------------------

\newcommand\artigo{a} 

\grau{Graduação}

\curso{Engenharia Elétrica} 
\newcommand\grauTitulo{Graduado em Engenharia Elétrica}  

\areadeconcentracao{Sistemas e Computação} 

%---------------------------------------------------------------------
% Informações adicionais (local, data e paginas)
%---------------------------------------------------------------------

\local{Rio de Janeiro} 
\data{DD}{Dezembro}{2024} % Exemplo: \data{21}{Março}{2016}

% ********************************************************************
% ********************************************************************
% Configurações de aparência do PDF final
% ********************************************************************
% ********************************************************************

% alterando o aspecto da cor azul
\definecolor{blue}{RGB}{41,5,195}
%\definecolor{apricot}{RGB}{251,206,177}

% informações do PDF
\hypersetup{
  unicode=false,
  pdftitle={\UERJtitulo},
  pdfauthor={\UERJautor},
  pdfsubject={\UERJpreambulo},
  pdfkeywords={PALAVRAS}{CHAVES}{\chaveA}{\chaveB}{\chaveC}{\chaveD},
  pdfproducer={\packagename}, % producer of the document
  pdfcreator={\UERJautor},
  colorlinks=true,            % false: boxed links; true: colored links
  linkcolor=black,            % color of internal links blue
  citecolor=black,            % color of links to bibliography blue
  filecolor=black,            % color of file links magenta
  urlcolor=black,
  bookmarksdepth=4,
  %backref=true,
  %pagebackref=true,
  %bookmarks=true,
}

% ********************************************************************
% ********************************************************************
% Início do documento
% ********************************************************************
% ********************************************************************
% ---
% compila o índice; se não for usar, comentar
% ---
\makeindex
% ********************************************************************
% ********************************************************************

\selectlanguage{brazil} % setando o idioma global para português


\begin{document}
% ----------------------------------------------------------
%% ELEMENTOS PRE-TEXTUAIS
% ----------------------------------------------------------
\frontmatter

\capa
\folhaderosto

% ----------------------------------------------------------
% Inserir a ficha catalográfica
% ----------------------------------------------------------
% ---
% A biblioteca deverá providenciar a ficha catalográfica. Salve a ficha no
% formato PDF. Use o nome do arquivo PDF como argumento do comando. 
% Exemplo: ficha catalográfica é o arquivo 'Ficha.pdf' na pasta "pre-textual"
%     \fichacatalografica{pre-textual/Ficha.pdf}
%
% Enquanto não possuir a ficha catalográfica, use o comando sem argumentos.
% ---
\fichacatalografica{pre-textual/Ficha.pdf}



% ----------------------------------------------------------
% Folha de aprovação
% ----------------------------------------------------------

%Após obter a assinatura dos membros da banca, comente as linhas a baixos e insira o pdf com as assinaturas na pasta "pre-textual"


% Ajustar o espaço entre as assinaturas abaixo
\newcommand{\spc}{0.25cm}

\begin{folhadeaprovacao}
	\vspace{\spc} % Espaço entre assinaturas
	\assinatura{Cargo Título Nome Completo}{Unidade - Instituição} %Exemplo
	\vspace{\spc} % Espaço entre assinaturas
	\assinatura{Cargo Título Nome Completo}{Unidade - Instituição}
	\vspace{\spc} % Espaço entre assinaturas
	\assinatura{Cargo Título Nome Completo}{Unidade - Instituição}
	\vspace{\spc} % Espaço entre assinaturas
	\assinatura{Cargo Título Nome Completo}{Unidade - Instituição}
\end{folhadeaprovacao}


% Após colocar o pdf com as assinaturas na pasta "pre-textual", comente todo o ambiente "folhadeaprovacao" acima, descomente a linha abaixo e insira o nome correto do arquivo pdf:
%\includepdf[pages=1]{pre-textual/ficha.pdf} %exemplo


% ----------------------------------------------------------
% Dedicatória
\pretextualchapter{Dedicatória}
\vfill
Eu dedico essa tese para uma pessoa muito especial.






% ----------------------------------------------------------
% Agradecimentos
\pretextualchapter{Agradecimentos}

Gostaria de expressar minha mais profunda gratidão a todos que, de alguma forma, contribuíram para a realização deste trabalho e para a conclusão da minha graduação.

Primeiramente, à minha mãe, Luzia Magalhães, pelo apoio incondicional ao longo de toda a minha trajetória acadêmica. Suas palavras de incentivo e seu entusiasmo com cada uma das minhas conquistas foram essenciais para que eu pudesse alcançar este momento tão significativo.

À Ketlin Rodrigues, por estar sempre ao meu lado, oferecendo suporte emocional e sendo uma presença constante de carinho e compreensão nos momentos mais desafiadores. Sua parceria foi indispensável nesta caminhada.

Aos meus amigos João Pedro Costa, Jorge dos Santos, Mayara Rodrigues, Rodrigo Tak-Ming, Gabriel Vidile e tantos outros que fizeram parte desta jornada, proporcionando momentos de alegria, suporte e colaboração inestimáveis.

Um agradecimento especial ao meu amigo Rafael Chrispim, que, mesmo não estando mais entre nós, deixou uma marca permanente em minha vida. Sua amizade, apoio e incentivo continuam a me inspirar e foram fundamentais para que eu me tornasse a pessoa que sou hoje.

À Growth Machine, empresa da qual sou sócio, pela flexibilidade e compreensão que me permitiram equilibrar meus compromissos profissionais e acadêmicos, proporcionando o espaço necessário para que eu pudesse crescer em ambos os campos.

À UERJ, ao Departamento de Sistemas e Computação e às minhas orientadoras, Rafaela e Cristiana, pela orientação e pelo suporte acadêmico ao longo deste percurso. A dedicação de vocês foi essencial para a realização deste trabalho e para o meu desenvolvimento como profissional.

A todos vocês, meu mais sincero muito obrigado. Cada um, de sua maneira, contribuiu para que eu chegasse até aqui e para que eu me tornasse o profissional que sou hoje. Este marco é também um reflexo de tudo que aprendi e vivenciei ao lado de vocês.


% ----------------------------------------------------------
% Epigrafe (opcional)
\pretextualchapter{}
\vfill
\begin{flushright}
	Feliz, feliz, feliz... Estou tão feliz\\
	\textit{Uma criança feliz}
\end{flushright}










% ----------------------------------------------------------
%% RESUMO
% se não for usar a quarta palavra chave, deixar o campo vazio: {}
\palavraschaves{Escalabilidade}
{Computação em nuvem}
{Aplicações em tempo real}
{WebSockets}


\pretextualchapter{Resumo}
\referencia % linha em branco depois

Desde a popularização da internet, a escalabilidade em aplicações globais tem se tornado um desafio crescente, especialmente com a popularização de tecnologias que fornecem interações em tempo real entre usuários. O presente trabalho explora os desafios e soluções de escalabilidade nessas aplicações, utilizando como estudo de caso a plataforma Codeboard UERJ. Desenvolvida para o ensino remoto, a Codeboard oferece um ambiente de programação colaborativa em tempo real, permitindo interações simultâneas entre professores e alunos. O estudo aborda conceitos fundamentais de desenvolvimento web, computação em nuvem e comunicação em tempo real, bem como a implementação de práticas avançadas de escalabilidade horizontal e vertical, balanceamento de carga e tolerância a falhas. A infraestrutura cloud foi projetada para lidar com altas demandas, empregando tecnologias como WebSockets, MongoDB, Redis e Auto Scaling. Os testes realizados demonstraram a eficácia das estratégias implementadas, assegurando desempenho consistente e confiabilidade mesmo sob condições adversas. Este trabalho contribui para o avanço da engenharia de sistemas distribuídos e serve como referência prática para desenvolvedores e arquitetos de software interessados em soluções escaláveis e com alta disponibilidade.


\imprimirchaves % linha em branco antes




% ----------------------------------------------------------
% Abstract
\begin{otherlanguage}{english}
\keywords{Scalability}
{Real-time Applications}
{Cloud Computing}
{WebSockets}


\pretextualchapter{Abstract}
\reference % linha em branco depois

Since the popularization of the internet, scalability in global applications has become an increasingly significant challenge, especially with the rise of technologies that enable real-time interactions between users. This paper explores the challenges and solutions related to scalability in such applications, using the Codeboard UERJ platform as a case study. Developed for remote learning, Codeboard provides a collaborative real-time programming environment, allowing simultaneous interactions between teachers and students. The study addresses fundamental concepts of web development, cloud computing, and real-time communication, as well as the implementation of advanced practices in horizontal and vertical scalability, load balancing, and fault tolerance. The cloud infrastructure was designed to handle high demands, leveraging technologies such as WebSockets, MongoDB, Redis, and Auto Scaling. The tests conducted demonstrated the effectiveness of the implemented strategies, ensuring consistent performance and reliability even under adverse conditions. This work contributes to the advancement of distributed systems engineering and serves as a practical reference for developers and software architects interested in scalable and highly available solutions.

\printkeys % linha em branco antes


\end{otherlanguage}



% ----------------------------------------------------------
% Listas de ilustrações, tabelas e algoritmos
% ----------------------------------------------------------
\listadefiguras
\listadetabelas
%\listadealgoritmos

% ----------------------------------------------------------
% Lista de abreviaturas, siglas e símbolos	
% \pretextualchapter{Lista de abreviaturas e siglas}
% ---
\abreviatura{CITT}{Técnica da Transformada integral Clássica}
\abreviatura{GITT}{Técnica da Transformada integral Generalizada}
\abreviatura{MVF}{Método de Volumes Finitos}








% \pretextualchapter{Lista de símbolos}
% ---
\simbolo{t}{Tempo}
\simbolo{L}{Dimensão na direção $x$}
\simbolo{H}{Dimensão na direção $y$}
\simbolo{\rho}{Massa específica}
\simbolo{\mu}{Viscosidade dinâmica}
\simbolo{\nu}{Viscosidade cinemática}








% ----------------------------------------------------------

\sumario

% ----------------------------------------------------------
%% ELEMENTOS TEXTUAIS
% ----------------------------------------------------------
\mainmatter

\chapter*{Introdução}

Na era digital em que vivemos, as aplicações em tempo real tornaram-se um componente crucial que conecta a sociedade moderna. Seja em redes sociais, jogos online ou plataformas
de streaming, a busca por experiências interativas e instantâneas é constante e, durante a pandemia de COVID-19, a necessidade dessas plataformas digitais tornou-se ainda mais evidente, especialmente no campo educacional \cite{impact-covid19-teaching-learning}. Com o fechamento das instituições de ensino e a transição para o formato remoto, tecnologias capazes de oferecer colaboração em tempo real ganharam destaque como elementos essenciais para manter a continuidade do aprendizado.

A escalabilidade, nesse cenário, se mostra como um dos maiores desafios técnicos para o desenvolvimento de sistemas globais, pois é necessário atender a um número crescente de usuários e à crescente complexidade das interações em tempo real. Soluções robustas em computação em nuvem e estratégias de paralelismo têm se mostrado indispensáveis para superar esses obstáculos e garantir desempenho e confiabilidade mesmo em condições de alta demanda.

Este projeto de graduação tem como objetivo explorar as nuances da escalabilidade em aplicações globais em tempo real, com foco na análise e desenvolvimento da plataforma Codeboard UERJ. Voltada para o ensino de programação, a plataforma permite a colaboração simultânea de professores e alunos por meio de um ambiente de codificação interativo e dinâmico. Ao longo do trabalho, serão investigadas práticas e estratégias que garantem uma infraestrutura capaz de se adaptar à demanda, mantendo um alto nível de desempenho e confiabilidade.

Ao documentar os desafios enfrentados e as soluções implementadas, este estudo busca contribuir para a compreensão e o avanço da engenharia de sistemas em larga escala. Além disso, espera-se que os insights obtidos sirvam como referência prática para desenvolvedores, arquitetos de sistemas e tomadores de decisão, promovendo a excelência na entrega de serviços em tempo real e fortalecendo a conexão entre tecnologia e educação.

Nos capítulos que seguem, serão apresentados os fundamentos teóricos que embasam este trabalho, incluindo conceitos de desenvolvimento web, computação em nuvem e protocolos de comunicação. Em seguida, será detalhada a arquitetura e implementação da plataforma Codeboard UERJ, abordando as estratégias adotadas para garantir sua robustez e eficiência. Os capítulos posteriores trarão uma análise dos testes realizados, os resultados obtidos e as discussões sobre as soluções implementadas. Por fim, serão destacadas as conclusões alcançadas e as perspectivas para trabalhos futuros.
\section{Conceitos Básicos}
% 🆗 Revisado

Antes de aprofundar nos aspectos técnicos específicos deste trabalho, é fundamental entender os conceitos que sustentam o desenvolvimento de aplicações web e a computação em nuvem. Este capítulo apresenta de forma sucinta esses conceitos básicos, fornecendo uma base para a compreensão das tecnologias e práticas utilizadas na construção e operação de aplicações web modernas.

\subsection{Desenvolvimento Web}
% 🆗 Revisado

O desenvolvimento web é a área da engenharia de computação que se dedica à criação de aplicações e serviços acessíveis através da internet. Envolve a construção de sites, aplicações web e outras soluções online que interagem com usuários por meio de navegadores ou dispositivos conectados.

\subsubsection{Aplicações Web}
% 🆗 Revisado

% O que são aplicações web?
% Que tecnologias são utilizadas para desenvolver aplicações web? (HTML, CSS, JavaScript, etc.)

Aplicações web são programas ou sistemas desenvolvidos para serem executados em navegadores de internet, permitindo que usuários acessem funcionalidades e informações através da web. Diferentemente dos softwares tradicionais instalados localmente, as aplicações web podem ser acessadas de qualquer lugar com conexão à internet, facilitando a distribuição e atualização.

Para desenvolver aplicações web, utilizam-se diversas tecnologias que colaboram entre si:

\begin{itemize}
    \item \textbf{HTML (HyperText Markup Language)}: Linguagem de marcação responsável por estruturar o conteúdo da web.
    \item \textbf{CSS (Cascading Style Sheets)}: Linguagem de estilo utilizada para definir a aparência e o layout dos documentos HTML.
    \item \textbf{JavaScript}: Linguagem de programação que permite adicionar interatividade e dinamismo às páginas web.
\end{itemize}

Essas tecnologias constituem a base do desenvolvimento web front-end, proporcionando interfaces amigáveis e funcionais para os usuários.

\subsubsection{Front-end e Back-end}
% 🆗 Revisado

% O que é front-end?
% O que é back-end?
% Qual a diferença entre front-end e back-end?
% Como se comunicam? (via HTTP por meio de APIs, via WebSockets, etc.)

No desenvolvimento de aplicações web, a arquitetura é geralmente dividida em duas camadas principais:

\begin{itemize}
    \item \textbf{Front-end}: Refere-se à parte da aplicação que interage diretamente com o usuário. Inclui tudo o que o usuário vê e com o que interage no navegador, como layouts, botões e formulários. Tecnologias como HTML, CSS e JavaScript, além de frameworks como React.js, são comumente utilizadas.
    \item \textbf{Back-end}: É a camada de servidor da aplicação, responsável pelo processamento de dados, lógica de negócio e comunicação com bancos de dados. Tecnologias como Node.js, Go, Java e frameworks como Express.js são utilizadas para desenvolver o back-end.
\end{itemize}

A comunicação entre o front-end e o back-end ocorre por meio de protocolos como HTTP. O front-end envia requisições ao back-end, que processa os dados e retorna respostas, geralmente via APIs. Em aplicações em tempo real, tecnologias como WebSockets também são utilizadas para comunicação bidirecional persistente entre cliente e servidor.

\subsubsection{APIs}
% TODO: Revisar

% O que é uma API?
% Quais são os tipos de APIs? (REST, GraphQL, etc.)
% O que é REST?

Uma \emph{API (Application Programming Interface)} é um conjunto de definições e protocolos que permite que diferentes softwares se comuniquem entre si. No contexto das aplicações web, as APIs permitem que o front-end interaja com o back-end para realizar operações como obtenção de dados e execução de ações no servidor.

Existem vários tipos de APIs, sendo atualmente dois dos mais populares:

\begin{itemize}
    \item \textbf{REST (Representational State Transfer)}: Um estilo arquitetural que utiliza protocolos HTTP para facilitar a comunicação entre cliente e servidor, oferecendo uma interface uniforme para manipulação de recursos.
    \item \textbf{GraphQL}: Uma linguagem de consulta para APIs que permite solicitar exatamente os dados necessários, melhorando a eficiência das requisições.
\end{itemize}

O \emph{REST} é amplamente adotado devido à sua simplicidade e aderência aos padrões web, facilitando a integração entre diferentes sistemas.

\subsubsection{Node.js}
% TODO: Revisar

% O que é Node.js?
% Quais são as vantagens de utilizar Node.js para desenvolver aplicações web?

\emph{Node.js} é um ambiente de execução JavaScript voltado para o desenvolvimento do lado do servidor (back-end). Ele permite que os desenvolvedores utilizem JavaScript fora do ambiente do navegador, possibilitando a criação de aplicações escaláveis e de alto desempenho.

Vantagens de utilizar Node.js para desenvolver aplicações web:

\begin{itemize}
    \item \textbf{Modelo assíncrono e orientado a eventos}: Permite lidar eficientemente com operações de I/O, resultando em aplicações leves e escaláveis.
    \item \textbf{Comunidade ativa e ecossistema rico}: A disponibilidade de inúmeros módulos e bibliotecas através do NPM (Node Package Manager) acelera o desenvolvimento.
    \item \textbf{Unificação da linguagem}: Usar JavaScript no front-end e back-end simplifica o desenvolvimento e mantém a consistência do código.
\end{itemize}

\subsubsection{Bibliotecas e Frameworks}
% 🆗 Revisado

% O que são frameworks? E bibliotecas? Qual a diferença entre eles?
% O que é React.js?
% O que é Next.js?
% O que é Express.js?

Frameworks e bibliotecas são ferramentas essenciais no desenvolvimento de software, amplamente empregadas para acelerar a criação de aplicações.

Bibliotecas consistem em coleções de funções ou componentes reutilizáveis que os desenvolvedores incorporam em seus projetos para resolver problemas específicos. Ao contrário dos frameworks, as bibliotecas não impõem uma estrutura fixa ao projeto, proporcionando maior flexibilidade e controle ao desenvolvedor sobre como e quando utilizá-las. Um exemplo notável é o React.js, uma biblioteca focada na criação de interfaces de usuário. Ela permite o uso de componentes reutilizáveis e oferece mecanismos eficientes para o gerenciamento de estados.

Frameworks, por sua vez, fornecem uma estrutura completa para o desenvolvimento de aplicações. Eles estabelecem uma arquitetura definida e um fluxo de trabalho consistente, além de integrarem um conjunto robusto de ferramentas para tarefas como roteamento, renderização e manipulação de requisições. Entre os frameworks mais populares, destacam-se:
\begin{itemize}
    \item \textbf{Next.js}: Um framework baseado em React que introduz funcionalidades avançadas, como renderização no lado do servidor (SSR) e geração de sites estáticos (SSG), otimizando o desempenho e melhorando a SEO.
    \item \textbf{Express.js}: Um framework minimalista para Node.js, projetado para simplificar o desenvolvimento de aplicações web e APIs. Ele facilita o gerenciamento de rotas, requisições e middlewares.
\end{itemize}

Tanto bibliotecas quanto frameworks desempenham um papel crucial no desenvolvimento de software. Essas ferramentas não apenas aceleram o processo de criação, mas também incentivam a adoção de boas práticas, contribuindo para a eficiência e qualidade dos projetos.


\subsection{Protocolos de Comunicação}
% 🆗 Revisado

Protocolos de comunicação são conjuntos de regras e convenções que determinam como os dados são transmitidos e recebidos entre sistemas. Esses protocolos desempenham um papel fundamental na garantia da interoperabilidade e integração entre os diversos componentes de um sistema distribuído, permitindo que diferentes tecnologias e dispositivos trabalhem juntos de maneira eficiente e coordenada.

\subsubsection{HTTP}
% 🆗 Revisado

O \emph{HTTP (Hypertext Transfer Protocol)} é o protocolo base da web, utilizado para a comunicação entre clientes (como navegadores) e servidores. Ele define as regras para formatação e transmissão de mensagens, bem como as ações que devem ser executadas em resposta a diferentes comandos.

O HTTP segue o modelo de requisição-resposta, no qual o cliente envia uma requisição ao servidor, que, por sua vez, processa a solicitação e retorna uma resposta. Tanto as requisições quanto as respostas são compostas por cabeçalhos, que contêm informações importantes, e, opcionalmente, por um corpo que transporta dados adicionais.

Entre os métodos HTTP mais utilizados, destacam-se:

\begin{itemize}
    \item \textbf{GET}: Utilizado para solicitar a representação de um recurso sem modificar seus dados.
    \item \textbf{POST}: Envia dados ao servidor para serem processados, frequentemente utilizado em formulários e envio de informações.
    \item \textbf{PUT}: Atualiza ou substitui a representação de um recurso existente.
    \item \textbf{DELETE}: Remove um recurso especificado.
\end{itemize}

O HTTP é fundamental para o funcionamento da web, permitindo a interação entre clientes e servidores de forma padronizada e eficiente.

\subsubsection{WebSockets}
% TODO: Revisar

\emph{WebSocket} é um protocolo que permite comunicação bidirecional e em tempo real entre cliente e servidor através de uma única conexão TCP. Diferentemente do HTTP, que é baseada em requisições, os WebSockets mantêm a conexão ativa, permitindo troca contínua de dados, ideal para aplicações que exigem atualização em tempo real.


\subsection{Banco de Dados}
% TODO: Escrever Intro

\subsubsection{SQL}
% TODO: Revisar

\emph{SQL (Structured Query Language)} é a linguagem padrão para gerenciamento de bancos de dados relacionais. Bancos de dados SQL armazenam dados em tabelas com esquemas definidos, permitindo consultas complexas e mantendo a consistência dos dados. Alguns dos bancos de dados SQL mais populares incluem:

\begin{itemize}
    \item \textbf{MySQL}: Um sistema de gerenciamento de banco de dados relacional de código aberto amplamente utilizado.
    \item \textbf{PostgreSQL}: Um sistema de gerenciamento de banco de dados relacional avançado, conhecido por sua robustez e recursos avançados.
    \item \textbf{SQLite}: Um banco de dados SQL embutido que não requer um servidor separado, adequado para aplicações de pequeno porte.
\end{itemize}

\subsubsection{NoSQL}
% TODO: Revisar
% O que é NoSQL?
% Quais são os tipos de bancos de dados NoSQL? (Documentos, Chave-Valor, Colunas, Grafos)

\emph{NoSQL} refere-se a uma categoria de sistemas de gerenciamento de banco de dados que não se baseiam no modelo relacional tradicional. Eles são projetados para lidar com grandes volumes de dados distribuídos e não estruturados.

Tipos de bancos de dados NoSQL:

\begin{itemize}
    \item \textbf{Documentos}: Armazenam dados como documentos semelhantes a JSON (e.g., MongoDB).
    \item \textbf{Chave-Valor}: Armazenam pares simples de chave e valor (e.g., Redis).
    \item \textbf{Colunas}: Organizam dados em colunas para leitura e escrita eficientes (e.g., Cassandra).
    \item \textbf{Grafos}: Focam em relações entre dados, utilizando nós e arestas (e.g., Neo4j).
\end{itemize}


\subsubsection{MongoDB}
% 🆗 Revisado

\emph{MongoDB} é um banco de dados NoSQL orientado a documentos que armazena os dados no formato BSON (uma extensão binária do JSON). Ele proporciona grande flexibilidade na modelagem de dados, sendo especialmente adequado para aplicações que demandam alta escalabilidade, desempenho e a capacidade de lidar com estruturas de dados dinâmicas ou não estruturadas.

\subsubsection{Redis}
% 🆗 Revisado

\emph{Redis} é um banco de dados em memória do tipo chave-valor, amplamente utilizado como banco de dados, cache e broker de mensagens. Ele oferece suporte a estruturas de dados avançadas, como strings, hashes, listas e conjuntos ordenados, garantindo alta performance e baixa latência, características que o tornam ideal para aplicações que exigem respostas rápidas e processamento eficiente.


\subsubsection{ORMs e ODMs}
% 🆗 Revisado

\emph{ORM (Object-Relational Mapping)} e \emph{ODM (Object-Document Mapping)} são técnicas que permitem mapear objetos do código para estruturas de bancos de dados.

\begin{itemize}
    \item \textbf{ORMs}: Facilitam a interação com bancos de dados relacionais, permitindo trabalhar com objetos em vez de escrever SQL (e.g., Sequelize para Node.js).
    \item \textbf{ODMs}: Realizam a mesma função para bancos de dados NoSQL orientados a documentos (e.g., Mongoose para MongoDB).
\end{itemize}

Essas ferramentas simplificam o desenvolvimento e aumentam a produtividade.

\subsection{Computação em Nuvem}
% 🆗 Revisado

% O que é computação em nuvem?
% Quais são os tipos de serviços de computação em nuvem? (IaaS, PaaS, SaaS)
% Quais são os principais provedores de serviços de computação em nuvem? (AWS, Google Cloud, Azure, etc.)

A computação em nuvem refere-se à entrega de serviços de computação por meio da internet, permitindo o acesso a recursos flexíveis e escaláveis sob demanda. Essa abordagem elimina a necessidade de investir em infraestrutura física, oferecendo aos usuários maior agilidade e eficiência operacional.

Tipos de serviços de computação em nuvem:

\begin{itemize}
    \item \textbf{IaaS (Infrastructure as a Service)}: Fornece infraestrutura básica, como servidores, armazenamento e redes, permitindo que os usuários configurem e gerenciem seus próprios ambientes de TI. (e.g., Amazon EC2, Google Compute Engine)
    \item \textbf{PaaS (Platform as a Service)}: Oferece plataformas de desenvolvimento completas, permitindo que os usuários criem, implantem e gerenciem aplicações sem se preocupar com a infraestrutura subjacente. (e.g., Vercel, Heroku)
    \item \textbf{SaaS (Software as a Service)}: Disponibiliza software e aplicativos prontos para uso diretamente pela internet, sem a necessidade de instalação ou manutenção local. (e.g., Google Drive, Salesforce)
\end{itemize}

Entre os principais provedores de serviços de computação em nuvem, destacam-se a \emph{Amazon Web Services (AWS)}, a \emph{Google Cloud Platform} e a \emph{Microsoft Azure}.

\subsubsection{Amazon Web Services (AWS)}
% TODO: Revisar
% O que é AWS?
% Quais são os principais serviços da AWS? (EC2, S3, RDS, etc.)

\emph{AWS} é uma plataforma de serviços em nuvem oferecida pela Amazon, proporcionando infraestrutura, poder computacional, armazenamento, banco de dados e serviços de distribuição de conteúdo em escala global.

Principais serviços da AWS:

\begin{itemize}
    \item \textbf{EC2 (Elastic Compute Cloud)}: Serviço de computação que permite criar instâncias de servidores virtuais na nuvem.
    \item \textbf{S3 (Simple Storage Service)}: Serviço de armazenamento de objetos altamente escalável e seguro.
    \item \textbf{RDS (Relational Database Service)}: Serviço gerenciado para bancos de dados relacionais.
\end{itemize}

\subsubsection{Instâncias EC2}
% TODO: Revisar
% O que é uma instância EC2?
% O que são tipos de instâncias EC2? (T2, M5, C5, etc.)

Uma instância EC2 é uma máquina virtual escalável que oferece capacidade computacional na nuvem. As instâncias variam em tamanho, poder computacional e recursos de memória, atendendo a diferentes requisitos.

Tipos de instâncias EC2:

\begin{itemize}
    \item \textbf{T2/T3}: Instâncias de uso geral com capacidade de "burst", ideais para cargas de trabalho com uso variável de CPU.
    \item \textbf{M5}: Instâncias de uso geral equilibradas em termos de computação, memória e recursos de rede.
    \item \textbf{C5}: Instâncias otimizadas para computação, adequadas para aplicações que requerem alto desempenho de CPU.
\end{itemize}

\subsubsection{Escalabilidade}
% TODO: Revisar
% O que é escalabilidade vertical?
% Como funciona a escalabilidade vertical? Quais são seus desafios?
% Quais são as vantagens da escalabilidade vertical?

% O que é escalabilidade horizontal?
% Como funciona a escalabilidade horizontal? Quais são seus desafios?
% Quais são as vantagens da escalabilidade horizontal?

Escalabilidade é a capacidade de um sistema de lidar com um aumento de carga adicionando recursos.

Escalabilidade vertical envolve adicionar mais recursos (CPU, memória) a uma única máquina. É simples de implementar, mas limitada pela capacidade máxima do hardware.

\begin{itemize}
    \item \textbf{Funcionamento}: Atualiza-se o hardware da máquina existente.
    \item \textbf{Desafios}: Limites físicos de expansão e possibilidade de downtime durante atualizações.
    \item \textbf{Vantagens}: Simplicidade e menor complexidade de gerenciamento.
\end{itemize}

Escalabilidade horizontal consiste em adicionar mais máquinas ao sistema, distribuindo a carga entre elas.

\begin{itemize}
    \item \textbf{Funcionamento}: Implementação de múltiplas instâncias que trabalham em paralelo.
    \item \textbf{Desafios}: Necessidade de balanceamento de carga e sincronização entre servidores.
    \item \textbf{Vantagens}: Maior resiliência e capacidade de expansão praticamente ilimitada.
\end{itemize}

\subsubsection{Balanceamento de Carga}
% TODO: Revisar
% O que é balanceamento de carga?
% Quais são os tipos de balanceamento de carga? (Round Robin, Least Connections, etc.)
% Quais são os serviços de balanceamento de carga na AWS? (ELB, ALB, NLB)

Balanceamento de carga é a distribuição uniforme do tráfego de rede ou processamento entre vários servidores.

Tipos de balanceamento de carga:

\begin{itemize}
    \item \textbf{Round Robin}: As requisições são distribuídas sequencialmente entre os servidores.
    \item \textbf{Least Connections}: As requisições são direcionadas ao servidor com o menor número de conexões ativas.
\end{itemize}

Serviços de balanceamento de carga na AWS:

\begin{itemize}
    \item \textbf{ELB (Elastic Load Balancer)}: Distribui automaticamente o tráfego de entrada entre múltiplas instâncias EC2.
    \item \textbf{ALB (Application Load Balancer)}: Opera na camada de aplicação (HTTP/HTTPS) e proporciona balanceamento de carga avançado.
    \item \textbf{NLB (Network Load Balancer)}: Opera na camada de transporte (TCP/UDP) e é capaz de lidar com tráfego de rede de alto desempenho.
\end{itemize}

\subsubsection{Mensageria}
% TODO: Revisar
% O que é mensageria?
% O que é um serviço de mensageria?
% Quais são os tipos de mensageria? (Fila, Tópico, etc.)
% O que é Pub/Sub?

Mensageria é a troca de mensagens entre sistemas ou componentes de software, permitindo comunicação assíncrona e desacoplada.

Tipos de mensageria:

\begin{itemize}
    \item \textbf{Fila}: As mensagens são armazenadas e processadas em ordem, garantindo que cada mensagem seja consumida por um único consumidor.
    \item \textbf{Tópico}: As mensagens são publicadas em tópicos e entregues a múltiplos assinantes interessados.
\end{itemize}

O \emph{Pub/Sub (Publish/Subscribe)} é um modelo onde os publishers enviam mensagens a tópicos, e os subscribers recebem as mensagens dos tópicos aos quais estão inscritos, permitindo comunicação um para muitos.

\subsubsection{DNS}
% TODO: Revisar
% O que é DNS?
% O que é um domínio?
% Qual serviço de DNS é utilizado na AWS? (Route 53)

\emph{DNS (Domain Name System)} é o sistema que traduz nomes de domínio legíveis por humanos (como www.exemplo.com) em endereços IP numéricos.

\begin{itemize}
    \item \textbf{Domínio}: É o nome exclusivo que identifica um site na internet.
\end{itemize}

Na AWS, o serviço de DNS é o \textbf{Route 53}, que oferece registro de domínios, resolução de nomes DNS e verificação da integridade de serviços.

\subsubsection{IaC e Terraform}
% TODO: Revisar
% O que é IaC?
% O que é Terraform?
% Como IaC beneficia o desenvolvimento de aplicações web?

\emph{textbf}{IaC (Infrastructure as Code)} é a prática de gerenciar e provisionar a infraestrutura de TI através de arquivos de definição de código legíveis por máquina, em vez de configuração manual.

\begin{itemize}
    \item \textbf{Terraform}: Uma ferramenta de IaC que permite definir e provisionar infraestrutura em múltiplos provedores de nuvem usando arquivos de configuração.
\end{itemize}

Benefícios da IaC no desenvolvimento de aplicações web:

\begin{itemize}
    \item \textbf{Automação}: Reduz erros humanos e aumenta a eficiência.
    \item \textbf{Consistência}: Proporciona ambientes replicáveis e padronizados.
    \item \textbf{Colaboração}: Facilita o versionamento e mudanças controladas na infraestrutura.
    \item \textbf{Escalabilidade}: Permite provisionar recursos de forma dinâmica conforme a demanda.
\end{itemize}

\subsection{Segurança}
% TODO: Revisar

A segurança é um aspecto fundamental no desenvolvimento e operação de aplicações web. Ela engloba medidas e práticas destinadas a proteger sistemas, dados e usuários contra acessos não autorizados, ataques e outras ameaças.


\subsubsection{Criptografia}
% TODO: Revisar
% O que é criptografia?
% O que é criptografia em repouso?
% O que é criptografia em trânsito?

Criptografia é a técnica de proteger informações através da codificação, tornando-as ininteligíveis para usuários não autorizados. Para isso são utilizados algoritmos e chaves criptograficas capazes de cifrar e decifrar os dados. Existem dois tipos principais de criptografia comumente empregados em aplicações web:

\begin{itemize}
    \item \textbf{Criptografia em repouso}: Protege dados armazenados em mídias físicas, como discos e bancos de dados.
    \item \textbf{Criptografia em trânsito}: Protege dados enquanto são transmitidos entre sistemas, evitando interceptações e ataques man-in-the-middle.
\end{itemize}

\subsubsection{Certificados SSL}
% TODO: Revisar
% O que é um certificado SSL?
% O que é HTTPS?

Um \emph{certificado SSL (Secure Sockets Layer)} é um arquivo digital que autentica a identidade de um site e habilita uma conexão criptografada. O uso de SSL resulta no protocolo \emph{HTTPS (HyperText Transfer Protocol Secure)}, garantindo que os dados transmitidos entre o servidor e o cliente sejam seguros.

\subsubsection{SSH}
% TODO: Revisar
% O que é SSH?

\emph{SSH (Secure Shell)} é um protocolo de rede que permite o acesso remoto seguro a máquinas, criptografando a sessão para proteger contra interceptações e ataques. É comumente utilizado para acesso a servidores, execução de comandos remotos e transferências seguras de arquivos.

\subsubsection{Firewalls}
% TODO: Revisar
% O que é um firewall?
% O que é um grupo de segurança na AWS? Como funciona?


Um \emph{firewall} é um sistema de segurança que controla o tráfego de rede, permitindo ou bloqueando comunicações com base em regras de segurança predefinidas.

Na AWS, os grupos de segurança funcionam como firewalls virtuais para recursos como instâncias EC2, controlando o tráfego de entrada e saída.

\begin{itemize}
    \item \textbf{Funcionamento}: As regras dos grupos de segurança especificam protocolos, portas e origens/destinos permitidos, protegendo as instâncias de tráfego indesejado.
\end{itemize}

Essas medidas de segurança são essenciais para proteger aplicações web contra ameaças e vulnerabilidades.










\include{chapters/3-codeboard.tex}
\include{chapters/4-cloud.tex}

\section{Testes e Resultados}

TODO: Importância dos testes voltados à infraestrutura.

TODO: Objetivo do capítulo: validar a escalabilidade, tolerância a falhas, desempenho e segurança da plataforma Codeboard UERJ.

\subsection{Testes de Escalabilidade}

\subsubsection{Escalabilidade Horizontal}

\subsubsection{Escalabilidade Vertical}

\subsection{Testes de Tolerância a Falhas}

\subsubsection{Falhas em Servidores}

\subsubsection{Recuperação Automática}

\subsubsection{Falha em Regiões Geográficas}

\subsection{Testes de Desempenho}

\subsubsection{Tempo de Resposta}

\subsubsection{Desempenho do Redis}

\subsubsection{Desempenho do MongoDB}

\subsection{Testes de Segurança}

\subsubsection{Testes de Firewall}

\subsubsection{Ataques de DDoS}

\subsubsection{Criptografia}

\subsection{Testes de Monitoramento}

\subsubsection{Monitoramento de Recursos}

\subsubsection{Alertas de Incidentes}




\backmatter

%=====================================================================
% Referencias via BibTeX
%=====================================================================
\citeoption{abnt-options4}
\bibliography{references.bib}


%=====================================================================
\end{document}