% se não for usar a quarta palavra chave, deixar o campo vazio: {}
\palavraschaves{Escalabilidade}
{Computação em nuvem}
{Aplicações em tempo real}
{WebSockets}


\pretextualchapter{Resumo}
\referencia % linha em branco depois

Desde a popularização da internet, a escalabilidade em aplicações globais tem se tornado um desafio crescente, especialmente com a popularização de tecnologias que fornecem interações em tempo real entre usuários. O presente trabalho explora os desafios e soluções de escalabilidade nessas aplicações, utilizando como estudo de caso a plataforma Codeboard UERJ. Desenvolvida para o ensino remoto, a Codeboard oferece um ambiente de programação colaborativa em tempo real, permitindo interações simultâneas entre professores e alunos. O estudo aborda conceitos fundamentais de desenvolvimento web, computação em nuvem e comunicação em tempo real, bem como a implementação de práticas avançadas de escalabilidade horizontal e vertical, balanceamento de carga e tolerância a falhas. A infraestrutura cloud foi projetada para lidar com altas demandas, empregando tecnologias como WebSockets, MongoDB, Redis e Auto Scaling. Os testes realizados demonstraram a eficácia das estratégias implementadas, assegurando desempenho consistente e confiabilidade mesmo sob condições adversas. Este trabalho contribui para o avanço da engenharia de sistemas distribuídos e serve como referência prática para desenvolvedores e arquitetos de software interessados em soluções escaláveis e com alta disponibilidade.


\imprimirchaves % linha em branco antes


