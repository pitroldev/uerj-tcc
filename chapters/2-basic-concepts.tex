\section{Conceitos Básicos}
% 🆗 Revisado

Antes de aprofundar nos aspectos técnicos específicos deste trabalho, é fundamental entender os conceitos que sustentam o desenvolvimento de aplicações web e a computação em nuvem. Este capítulo apresenta de forma sucinta esses conceitos básicos, fornecendo uma base para a compreensão das tecnologias e práticas utilizadas na construção e operação de aplicações web modernas.

\subsection{Desenvolvimento Web}
% 🆗 Revisado

O desenvolvimento web é a área da engenharia de computação que se dedica à criação de aplicações e serviços acessíveis através da internet. Envolve a construção de sites, aplicações web e outras soluções online que interagem com usuários por meio de navegadores ou dispositivos conectados.

\subsubsection{Aplicações Web}
% 🆗 Revisado e Citado

% O que são aplicações web?
% Que tecnologias são utilizadas para desenvolver aplicações web? (HTML, CSS, JavaScript, etc.)

Aplicações web são programas ou sistemas desenvolvidos para serem executados em navegadores de internet, permitindo que usuários acessem funcionalidades e informações através da web. Diferentemente dos softwares tradicionais instalados localmente, as aplicações web podem ser acessadas de qualquer lugar com conexão à internet, facilitando a distribuição e atualização. \cite{web-app}

Para desenvolver aplicações web, utilizam-se diversas tecnologias que colaboram entre si:

\begin{itemize}
    \item \textbf{HTML (HyperText Markup Language)}: Linguagem de marcação responsável por estruturar o conteúdo da web.
    \item \textbf{CSS (Cascading Style Sheets)}: Linguagem de estilo utilizada para definir a aparência e o layout dos documentos HTML.
    \item \textbf{JavaScript}: Linguagem de programação que permite adicionar interatividade e dinamismo às páginas web.
\end{itemize}

Essas tecnologias constituem a base do desenvolvimento web front-end, proporcionando interfaces amigáveis e funcionais para os usuários.

\subsubsection{Front-end e Back-end}
% 🆗 Revisado e Citado

% O que é front-end?
% O que é back-end?
% Qual a diferença entre front-end e back-end?
% Como se comunicam? (via HTTP por meio de APIs, via WebSockets, etc.)

No desenvolvimento de aplicações web, a arquitetura é geralmente dividida em duas camadas principais\cite{web-app}:

\begin{itemize}
    \item \textbf{Front-end}: Refere-se à parte da aplicação que interage diretamente com o usuário. Inclui tudo o que o usuário vê e com o que interage no navegador, como layouts, botões e formulários. Tecnologias como HTML, CSS e JavaScript, além de frameworks como React.js, são comumente utilizadas.
    \item \textbf{Back-end}: É a camada de servidor da aplicação, responsável pelo processamento de dados, lógica de negócio e comunicação com bancos de dados. Tecnologias como Node.js, Go, Java e frameworks como Express.js são utilizadas para desenvolver o back-end.
\end{itemize}

A comunicação entre o front-end e o back-end normalmente ocorre por meio de protocolos como o HTTP. O front-end envia requisições, geralmente via APIs\cite{web-app}, ao back-end, que processa os dados e retorna respostas. Em aplicações em tempo real, tecnologias como WebSockets também são utilizadas para comunicação bidirecional\cite{ws-standard} persistente entre cliente e servidor.

\subsubsection{APIs}
% 🆗 Revisado e Citado

% O que é uma API?
% Quais são os tipos de APIs? (REST, GraphQL, etc.)
% O que é REST?

Uma \emph{API (Application Programming Interface)} é um conjunto de definições e protocolos que permite a comunicação entre diferentes softwares. No contexto de aplicações web, as APIs desempenham um papel crucial ao possibilitar a interação entre o front-end e o back-end, permitindo operações como a obtenção de dados e a execução de ações no servidor. \cite{aws-api-types}

Existem diversos tipos de APIs, cada um com suas características e finalidades específicas. Alguns dos mais comuns incluem:

\begin{itemize}
    \item \textbf{REST (Representational State Transfer)}: Um estilo arquitetural baseado no protocolo HTTP que oferece uma interface uniforme para manipulação de recursos. É amplamente adotado por sua simplicidade e aderência aos padrões da web, facilitando a integração entre sistemas.
    \item \textbf{WebSocket}: Um protocolo de comunicação bidirecional em tempo real que permite a troca de mensagens entre cliente e servidor. É ideal para aplicações que demandam comunicação assíncrona e interativa.
    \item \textbf{RPC (Remote Procedure Calls)}: Um framework de comunicação remota de alto desempenho desenvolvido pela Google, onde o cliente chama funções remotas como se fossem locais.
    \item \textbf{SOAP (Simple Object Access Protocol)}: Um protocolo baseado em XML para troca de mensagens entre sistemas distribuídos, frequentemente utilizado em integrações corporativas.
\end{itemize}

Cada tipo de API possui suas próprias características e é adequado para diferentes cenários de uso. A escolha do tipo de API depende das necessidades específicas do projeto, como complexidade, desempenho e flexibilidade.


\subsubsection{Node.js}
% 🆗 Revisado

% O que é Node.js?
% Quais são as vantagens de utilizar Node.js para desenvolver aplicações web?

\emph{Node.js} é um ambiente de execução de JavaScript projetado para o desenvolvimento do lado do servidor (back-end). Ele permite que os desenvolvedores utilizem JavaScript fora do navegador, possibilitando a criação de aplicações escaláveis, de alto desempenho e adequadas para diferentes tipos de cenários, como APIs, sistemas em tempo real e microsserviços.

Construído sobre o motor V8 do Google Chrome, o Node.js oferece execução extremamente rápida para código JavaScript. Ele adota um modelo de programação assíncrono e orientado a eventos\cite{nodejs-about}, ideal para lidar com operações intensivas de entrada e saída, como comunicação com bancos de dados ou requisições HTTP.

Vantagens de utilizar Node.js para desenvolver aplicações web:

\begin{itemize}
    \item \textbf{Modelo assíncrono e orientado a eventos}: O modelo \emph{non-blocking} (não bloqueante) permite que o servidor lide com múltiplas conexões simultaneamente, tornando-o altamente eficiente e ideal para aplicações que demandam grande escalabilidade, como chats em tempo real e streaming de dados.
    \item \textbf{Comunidade ativa e ecossistema rico}: A comunidade do Node.js é uma das mais movimentadas no desenvolvimento de software\cite{size-programming-languages}, e o NPM (Node Package Manager) disponibiliza milhões de módulos e bibliotecas prontos para uso, agilizando o desenvolvimento e aumentando a produtividade.
    \item \textbf{Unificação da linguagem}: O uso do JavaScript tanto no front-end quanto no back-end elimina a necessidade de lidas com linguagens diferentes para cada camada da aplicação, promovendo maior consistência e redução de complexidade.
\end{itemize}

Além disso, o Node.js é frequentemente escolhido para projetos modernos devido à sua flexibilidade e capacidade de suportar arquiteturas orientadas a microsserviços, permitindo que equipes desenvolvam e mantenham aplicações complexas com maior eficiência.

\subsubsection{Bibliotecas e Frameworks}
% 🆗 Revisado

% O que são frameworks? E bibliotecas? Qual a diferença entre eles?
% O que é React.js?
% O que é Next.js?
% O que é Express.js?

Frameworks e bibliotecas são ferramentas essenciais no desenvolvimento de software, são amplamente utilizadas para acelerar a criação de aplicações.

Bibliotecas consistem em um conjunto de funções ou componentes reutilizáveis que os desenvolvedores incorporam em seus projetos para resolver problemas específicos. Ao contrário dos frameworks, as bibliotecas não impõem uma estrutura fixa ao projeto, proporcionando maior flexibilidade e controle ao desenvolvedor sobre como e quando utilizá-las\cite{libs-vs-frameworks}. Um exemplo notável é o React.js, uma biblioteca focada na criação de interfaces de usuário. Ela permite o uso de componentes reutilizáveis e oferece mecanismos eficientes para o gerenciamento de estados.

Frameworks, por sua vez, fornecem uma estrutura completa para o desenvolvimento de aplicações. Eles estabelecem uma arquitetura definida e um fluxo de trabalho consistente, além de integrarem um conjunto robusto de ferramentas para tarefas como roteamento, renderização e manipulação de requisições. Entre os frameworks mais populares, destacam-se:
\begin{itemize}
    \item \textbf{Next.js}: Um framework que se baseia na biblioteca \emph{React}\cite{what-is-nextjs} e introduz funcionalidades avançadas, como renderização no lado do servidor (SSR) e geração de sites estáticos (SSG), otimizando o desempenho e melhorando a SEO \emph{(Search Engine Optimization)}.
    \item \textbf{Express.js}: Um framework minimalista para Node.js, projetado para simplificar o desenvolvimento de aplicações web e APIs. Ele facilita o gerenciamento de rotas, requisições e middlewares.
\end{itemize}

Tanto bibliotecas quanto frameworks desempenham um papel crucial no desenvolvimento de software. Essas ferramentas não apenas aceleram o processo de criação, mas também incentivam a adoção de boas práticas, contribuindo para a eficiência e qualidade dos projetos.


\subsection{Protocolos de Comunicação}
% 🆗 Revisado

Protocolos de comunicação são conjuntos de regras e convenções que determinam como os dados são transmitidos e recebidos entre sistemas. Esses protocolos desempenham um papel fundamental na garantia da interoperabilidade e integração entre os diversos componentes de um sistema distribuído, permitindo que diferentes tecnologias e dispositivos trabalhem juntos de maneira eficiente e coordenada.

\subsubsection{HTTP}
% 🆗 Revisado

O \emph{HTTP (Hypertext Transfer Protocol)} é o protocolo base da web, utilizado para a comunicação entre clientes (como navegadores) e servidores a nível de aplicação\cite{http-protocol}. Ele define as regras para formatação e transmissão de mensagens, bem como as ações que devem ser executadas em resposta a diferentes comandos.

O HTTP segue o modelo de requisição-resposta, no qual o cliente envia uma requisição ao servidor, que, por sua vez, processa a solicitação e retorna uma resposta. Tanto as requisições quanto as respostas são compostas por cabeçalhos, que contêm informações importantes, e, opcionalmente, por um corpo que transporta dados adicionais.

Entre os métodos HTTP mais utilizados, destacam-se:

\begin{itemize}
    \item \textbf{GET}: Utilizado para solicitar a representação de um recurso sem modificar seus dados.
    \item \textbf{POST}: Envia dados ao servidor para serem processados, frequentemente utilizado em formulários e envio de informações.
    \item \textbf{PUT}: Atualiza ou substitui a representação de um recurso existente.
    \item \textbf{DELETE}: Remove um recurso especificado.
\end{itemize}

Atualmente, o HTTP é indispensável para o funcionamento da internet moderna, permitindo uma interação padronizada e eficiente entre clientes e servidores, além de atuar como base para o desenvolvimento de aplicações web.

\subsubsection{WebSocket}
% 🆗 Revisado

\emph{WebSocket} é um protocolo de comunicação que permite uma interação bidirecional e em tempo real entre cliente e servidor por meio de uma única conexão TCP. Ele foi projetado para superar as limitações do protocolo HTTP, que é baseado no modelo de requisição-resposta e não mantém uma conexão persistente.\cite{ws-standard}

Ao contrário do HTTP, onde o cliente precisa iniciar cada requisição para receber novos dados, os WebSockets mantêm a conexão ativa após o handshake inicial. Isso possibilita que tanto o cliente quanto o servidor enviem mensagens de forma independente e assíncrona, sem a necessidade de estabelecer novas conexões para cada troca de dados. 

Características principais do WebSocket:

\begin{itemize}
    \item \textbf{Comunicação em tempo real}: Permite atualizações instantâneas de dados entre cliente e servidor, ideal para aplicações que demandam respostas rápidas, como chats, jogos online e sistemas de monitoramento.
    \item \textbf{Baixa latência}: Por evitar a sobrecarga de abertura e fechamento de conexões repetidas, o WebSocket reduz significativamente a latência, tornando-o mais eficiente que alternativas baseadas em HTTP.
    \item \textbf{Conexão persistente}: Uma única conexão é mantida ativa durante toda a interação, reduzindo o consumo de recursos do servidor e melhorando a escalabilidade.
    \item \textbf{Flexibilidade no transporte de dados}: Oferece suporte tanto para mensagens em texto quanto em formato binário, atendendo a uma ampla variedade de casos de uso.
\end{itemize}

Os WebSockets transformaram a forma como aplicações modernas lidam com comunicação em tempo real, oferecendo uma solução eficiente e flexível para cenários que demandam alta interatividade e respostas imediatas. Sua capacidade de manter conexões persistentes com baixa latência torna o protocolo indispensável em sistemas que exigem troca contínua de dados, consolidando-o como uma escolha estratégica para o desenvolvimento de aplicações web modernas que utilizam comunicação em tempo real\cite{murley2021websocket}.

\subsection{Banco de Dados}

Banco de dados é um sistema de armazenamento de dados que permite a organização, gerenciamento e recuperação eficiente de informações. Eles são fundamentais para aplicações que precisam armazenar e acessar dados de forma estruturada e segura. Existem diversos tipos de bancos de dados, cada um com suas características e finalidades específicas que atendem a diferentes necessidades de armazenamento e manipulação de dados.


\subsubsection{SQL}
% 🆗 Revisado

\emph{SQL (Structured Query Language)} é a linguagem padrão para gerenciamento de bancos de dados relacionais\cite{sql-wiki}. Bancos de dados SQL organizam os dados em tabelas estruturadas com esquemas predefinidos, permitindo consultas complexas, manipulação eficiente de dados e garantindo consistência por meio de transações.

Os bancos de dados baseados em SQL são amplamente utilizados devido à sua capacidade de lidar com grandes volumes de informações e à aderência a padrões que favorecem a interoperabilidade entre diferentes sistemas.

Exemplos de bancos de dados SQL:

\begin{itemize}
    \item \textbf{MySQL}: Um sistema de gerenciamento de banco de dados relacional de código aberto amplamente utilizado, especialmente em aplicações web. Ele é conhecido por sua simplicidade, desempenho e grande comunidade de suporte.
    \item \textbf{PostgreSQL}: Reconhecido por sua robustez e recursos avançados, como suporte a tipos de dados personalizados, índices complexos e transações ACID. É ideal para aplicações críticas que demandam alta confiabilidade.
    \item \textbf{SQLite}: Um banco de dados leve e embutido, que não requer um servidor dedicado. Ele é bastante utilizado em dispositivos móveis e aplicativos locais devido à sua portabilidade e simplicidade.
\end{itemize}

Os bancos de dados SQL continuam sendo a espinha dorsal de muitos sistemas corporativos e de produção, graças à sua eficiência, consistência e maturidade no mercado.

\subsubsection{NoSQL}
% 🆗 Revisado

% O que é NoSQL?
% Quais são os tipos de bancos de dados NoSQL? (Documentos, Chave-Valor, Colunas, Grafos)

\emph{NoSQL} é um termo que engloba uma categoria de sistemas de gerenciamento de banco de dados que não seguem o modelo relacional tradicional\cite{nosql-vs-sql}. Esses bancos de dados são projetados para lidar com grandes volumes de dados distribuídos, altamente escaláveis e frequentemente não estruturados. Eles oferecem maior flexibilidade para modelagem de dados e são amplamente utilizados em aplicações modernas, como redes sociais, IoT e análise de big data.

Principais tipos de bancos de dados NoSQL:

\begin{itemize}
    \item \textbf{Documentos}: Armazenam dados em formato de documentos, frequentemente similares ao JSON, permitindo flexibilidade na estrutura dos dados. Exemplos: MongoDB, Couchbase.
    \item \textbf{Chave-Valor}: Funcionam como um dicionário simples, onde cada chave está associada a um valor. São ideais para casos de uso como caches e armazenamento de sessões. Exemplos: Redis, DynamoDB.
    \item \textbf{Colunas}: Organizam os dados em colunas em vez de linhas, otimizando operações de leitura e escrita em larga escala. São frequentemente utilizados em sistemas de análise. Exemplos: Cassandra, HBase.
    \item \textbf{Grafos}: Focados nas relações entre os dados, utilizam nós para representar entidades e arestas para as conexões entre elas. São ideais para aplicações como redes sociais, mecanismos de recomendação e análise de conexões. Exemplos: Neo4j, ArangoDB.
\end{itemize}

Os bancos de dados NoSQL se destacam por sua capacidade de lidar com dados dinâmicos e escalar horizontalmente, oferecendo soluções sob medida para desafios específicos que muitas vezes não podem ser abordados de forma eficiente por bancos de dados relacionais tradicionais.

\subsubsection{MongoDB}
% 🆗 Revisado

\emph{MongoDB} é um banco de dados NoSQL orientado a documentos que armazena os dados no formato BSON (uma extensão binária do JSON). Ele proporciona grande flexibilidade na modelagem de dados, sendo especialmente adequado para aplicações que demandam alta escalabilidade, desempenho e a capacidade de lidar com estruturas de dados dinâmicas ou não estruturadas.

\subsubsection{Redis}
% 🆗 Revisado

\emph{Redis} é um banco de dados em memória do tipo chave-valor, amplamente utilizado como banco de dados, cache e broker de mensagens. Ele oferece suporte a estruturas de dados avançadas, como strings, hashes, listas e conjuntos ordenados, garantindo alta performance e baixa latência, características que o tornam ideal para aplicações que exigem respostas rápidas e processamento eficiente.


\subsubsection{ORMs e ODMs}
% 🆗 Revisado

\emph{ORM (Object-Relational Mapping)} e \emph{ODM (Object-Document Mapping)} são técnicas que permitem mapear objetos do código-fonte para estruturas de bancos de dados relacionais e NoSQL, respectivamente. Elas têm como objetivo simplificar a interação com os bancos de dados, permitindo que os desenvolvedores trabalhem diretamente com a linguagem de programação utilizada no projeto, em vez de escrever consultas SQL ou comandos específicos para cada tipo de banco.\cite{orm-vs-odm}

\begin{itemize}
    \item \textbf{ORMs}: Facilitam a manipulação de dados em bancos de dados relacionais, abstraindo consultas SQL e permitindo o uso de objetos. Exemplo: Sequelize.
    \item \textbf{ODMs}: Executam uma função semelhante para bancos de dados NoSQL orientados a documentos, integrando objetos de forma transparente. Exemplo: Mongoose.
\end{itemize}

Essas ferramentas aumentam a produtividade ao abstrair a complexidade das operações de banco de dados, proporcionando uma camada intermediária que simplifica a manipulação de dados e facilita a integração entre a aplicação e o banco.

\subsection{Computação em Nuvem}
% 🆗 Revisado

% O que é computação em nuvem?
% Quais são os tipos de serviços de computação em nuvem? (IaaS, PaaS, SaaS)
% Quais são os principais provedores de serviços de computação em nuvem? (AWS, Google Cloud, Azure, etc.)

A computação em nuvem refere-se à entrega de serviços de computação por meio da internet, permitindo o acesso a recursos de maneira flexível, escalável e sob demanda\cite{what-is-cloud}. Essa abordagem elimina a necessidade de investir em infraestrutura física, oferecendo maior agilidade e eficiência operacional para empresas e indivíduos.

Tipos de serviços de computação em nuvem:

\begin{itemize}
    \item \textbf{IaaS (Infrastructure as a Service)}: Proporciona acesso a infraestrutura básica, como servidores, armazenamento e redes. Os usuários têm controle total sobre o ambiente, configurando e gerenciando conforme suas necessidades. Exemplos: Amazon EC2, Google Compute Engine.
    \item \textbf{PaaS (Platform as a Service)}: Oferece plataformas completas para desenvolvimento e implantação de aplicações, abstraindo a complexidade da infraestrutura subjacente. Ideal para acelerar o desenvolvimento e focar na aplicação. Exemplos: Vercel, Heroku.
    \item \textbf{SaaS (Software as a Service)}: Disponibiliza aplicativos prontos para uso por meio da internet, eliminando a necessidade de instalação ou manutenção local. É amplamente utilizado para produtividade, colaboração e gerenciamento de negócios. Exemplos: Google Drive, Salesforce.
\end{itemize}

Entre os principais provedores de serviços de computação em nuvem, destacam-se a \emph{Amazon Web Services (AWS)}, a \emph{Google Cloud Platform} e a \emph{Microsoft Azure}. Essas plataformas oferecem uma ampla gama de serviços e soluções que atendem às necessidades de empresas de diferentes tamanhos, desde \emph{startups} até grandes corporações.

\subsubsection{Amazon Web Services (AWS)}
% 🆗 Revisado

% O que é AWS?
% Quais são os principais serviços da AWS? (EC2, S3, RDS, etc.)

A \emph{Amazon Web Services (AWS)} é uma plataforma líder em computação em nuvem\cite{cloud-ranking}, oferecida pela Amazon. Ela fornece uma ampla gama de serviços, incluindo infraestrutura, poder computacional, armazenamento, banco de dados e distribuição de conteúdo, operando em escala global. A AWS é amplamente utilizada por empresas de diferentes tamanhos devido à sua flexibilidade, escalabilidade e confiabilidade.

Dentre os serviços mais populares da AWS, destacam-se:

\begin{itemize}
    \item \textbf{EC2 (Elastic Compute Cloud)}: Serviço de computação que permite criar instâncias de servidores virtuais na nuvem, oferecendo flexibilidade para dimensionar o poder computacional conforme necessário.
    \item \textbf{S3 (Simple Storage Service)}: Um serviço de armazenamento de objetos que combina alta escalabilidade, segurança e custo acessível, adequado para armazenamento de arquivos, backups e grandes volumes de dados.
    \item \textbf{RDS (Relational Database Service)}: Serviço gerenciado que simplifica a configuração, operação e escalabilidade de bancos de dados relacionais, suportando motores populares como MySQL, PostgreSQL, e SQL Server.
\end{itemize}

A AWS se destaca como uma solução estratégica para organizações que buscam inovação, escalabilidade e alta disponibilidade em suas operações na nuvem. Com uma vasta gama de serviços gerenciados e um ecossistema robusto, a plataforma permite que empresas de todos os portes implementem soluções personalizadas de forma eficiente e segura.

\subsubsection{Instâncias EC2}
% 🆗 Revisado

% O que é uma instância EC2?
% O que são tipos de instâncias EC2? (T2, M5, C5, etc.)

Uma instância EC2 (\emph{Elastic Compute Cloud}) é uma máquina virtual fornecida pela AWS que oferece capacidade computacional na nuvem. Essas instâncias permitem que usuários executem aplicações com flexibilidade e escalabilidade, sem a necessidade de gerenciar hardware físico. A AWS oferece uma ampla variedade de tipos de instâncias para atender às diferentes demandas de desempenho, custo e funcionalidade.

Os tipos de instâncias EC2 mais comuns incluem:

\begin{itemize}
    \item \textbf{T2/T3}: Instâncias de uso geral com capacidade de \emph{burst}, projetadas para cargas de trabalho com variação no uso de CPU, como pequenos servidores web e ambientes de desenvolvimento.
    \item \textbf{M5}: Instâncias de uso geral balanceadas, ideais para uma ampla gama de aplicações, como bancos de dados pequenos e servidores de médio porte.
    \item \textbf{C5}: Instâncias otimizadas para computação, desenvolvidas para tarefas que exigem alto desempenho de CPU, como simulações científicas e processamento intensivo de dados.
\end{itemize}

Cada tipo de instância oferece opções configuráveis em termos de número de vCPUs, memória e armazenamento, permitindo que os usuários escolham a melhor solução para suas necessidades específicas de computação.

\subsubsection{Escalabilidade}
% 🆗 Revisado

% O que é escalabilidade vertical?
% Como funciona a escalabilidade vertical? Quais são seus desafios?
% Quais são as vantagens da escalabilidade vertical?

% O que é escalabilidade horizontal?
% Como funciona a escalabilidade horizontal? Quais são seus desafios?
% Quais são as vantagens da escalabilidade horizontal?

Escalabilidade é a capacidade de um sistema de adaptar-se ao aumento da carga de trabalho, mantendo seu desempenho através da adição de recursos. Existem dois principais tipos de escalabilidade: vertical e horizontal\cite{what-is-scalability}.

A escalabilidade vertical consiste em melhorar uma única máquina adicionando mais recursos, como CPU, memória ou armazenamento\cite{what-is-scalability}. Essa abordagem é ideal para situações em que o objetivo é melhorar o desempenho sem alterar a arquitetura existente. Em sua implementação, o hardware do sistema é atualizado para lidar com maiores demandas. No entanto, esse modelo enfrenta limitações, como os limites físicos e técnicos do hardware disponível e a possibilidade de interrupções (\emph{downtime}) durante as atualizações. Apesar disso, a escalabilidade vertical se destaca pela simplicidade de sua implementação, uma vez que não exige mudanças significativas na aplicação ou na infraestrutura, o que também facilita o gerenciamento.

Por outro lado, a escalabilidade horizontal implica na adição de novas máquinas ao sistema, distribuindo a carga de trabalho entre elas\cite{what-is-scalability}. Esse modelo é mais adequado para aplicações que precisam processar grandes volumes de dados ou tráfego. Ele funciona ao replicar instâncias do sistema e utilizando balanceadores de carga para direcionar as solicitações de maneira eficiente. Apesar de suas vantagens, como maior capacidade de expansão, resiliência e tolerância a falhas (pois não depende de um único ponto), a escalabilidade horizontal apresenta desafios. Entre eles estão a necessidade de arquiteturas adequadas para distribuição de carga, sincronização de dados entre servidores e maior esforço de configuração e gerenciamento.

A decisão entre escalabilidade vertical e horizontal depende das características específicas da aplicação, do orçamento disponível e dos requisitos de desempenho e disponibilidade do sistema. Em muitos casos, é comum a combinação de ambas as abordagens para alcançar um equilíbrio ideal entre custo, eficiência e flexibilidade.

\subsubsection{Balanceamento de Carga}
% 🆗 Revisado

% O que é balanceamento de carga?
% Quais são os tipos de balanceamento de carga? (Round Robin, Least Connections, etc.)
% Quais são os serviços de balanceamento de carga na AWS? (ELB, ALB, NLB)

O balanceamento de carga é o processo de distribuir uniformemente o tráfego de rede ou o processamento entre vários servidores\cite{what-is-load-balancing}. Essa técnica é essencial para garantir a alta disponibilidade, melhorar o desempenho e evitar sobrecarga em um único servidor, assegurando que as requisições sejam atendidas de forma eficiente.

Os tipos mais comuns de algoritmos de balanceamento de carga são:

\begin{itemize}
    \item \textbf{Round Robin}: Distribui as requisições sequencialmente entre os servidores disponíveis, de forma cíclica.
    \item \textbf{Least Connections}: Direciona as requisições ao servidor que possui o menor número de conexões ativas no momento, otimizando o uso de recursos.
    \item \textbf{IP Hash}: Baseia a distribuição das requisições no endereço IP do cliente, garantindo que as solicitações do mesmo cliente sejam sempre direcionadas ao mesmo servidor.
    \item \textbf{Least Response Time}: Encaminha as requisições ao servidor que respondeu mais rapidamente às solicitações anteriores, visando otimizar o tempo de resposta.
\end{itemize}

Na AWS, os serviços de balanceamento de carga disponíveis incluem:

\begin{itemize}
    \item \textbf{ELB (Elastic Load Balancer)}: Serviço geral de balanceamento de carga que distribui automaticamente o tráfego entre múltiplas instâncias EC2, ajudando a manter a resiliência e a escalabilidade.
    \item \textbf{ALB (Application Load Balancer)}: Funciona na camada de aplicação (HTTP/HTTPS) e oferece funcionalidades avançadas, como roteamento baseado em URL, cabeçalhos e cookies.
    \item \textbf{NLB (Network Load Balancer)}: Opera na camada de transporte (TCP/UDP) e é otimizado para lidar com grandes volumes de tráfego de rede com latência extremamente baixa.
\end{itemize}

\subsubsection{Mensageria}
% 🆗 Revisado

% O que é mensageria?
% O que é um serviço de mensageria?
% Quais são os tipos de mensageria? (Fila, Tópico, etc.)
% O que é Pub/Sub?

Mensageria refere-se à troca de mensagens entre sistemas ou componentes de software, promovendo comunicação assíncrona e desacoplada. Essa abordagem é essencial em arquiteturas distribuídas, permitindo que diferentes partes de um sistema interajam de forma eficiente sem depender diretamente umas das outras.

Existem dois principais tipos de mensageria\cite{what-is-message-broker}. O primeiro é o modelo de fila, no qual as mensagens são armazenadas em uma estrutura sequencial e processadas em ordem (geralmente \emph{FIFO}, ou seja, \emph{"First In, First Out"}). Nesse modelo, cada mensagem é consumida por um único consumidor, sendo particularmente útil para a execução de tarefas assíncronas. O segundo tipo é o modelo de publicação em tópicos, conhecido como \emph{Publish/Subscribe}, onde as mensagens são publicadas em um canal denominado tópico e podem ser entregues a múltiplos assinantes interessados, permitindo comunicação de um para muitos.

O modelo \emph{Pub/Sub (Publish/Subscribe)} é amplamente utilizado em sistemas de mensageria. Nesse modelo, os publicadores (\emph{publishers}) enviam mensagens para tópicos sem conhecimento direto dos consumidores (\emph{subscribers}). Os consumidores, por sua vez, recebem as mensagens dos tópicos aos quais estão inscritos. Essa abordagem facilita a transmissão de eventos e notificações em tempo real, proporcionando flexibilidade e escalabilidade para sistemas que exigem comunicação eficiente e distribuída.

Mensageria desempenha um papel fundamental na construção de sistemas resilientes e escaláveis, oferecendo suporte para comunicação assíncrona e desacoplada, além de ser a base para arquiteturas orientadas a eventos.

\subsubsection{DNS}
% 🆗 Revisado

% O que é DNS?
% O que é um domínio?
% Qual serviço de DNS é utilizado na AWS? (Route 53)

O \emph{DNS (Domain Name System)} é um sistema essencial para a internet, responsável por traduzir nomes de domínio legíveis por humanos (como www.uerj.br) em endereços IP numéricos utilizados por computadores para localizar e acessar recursos online\cite{what-is-dns}. Sem o DNS, seria necessário memorizar complexos endereços IP para acessar sites e serviços.

Um domínio é o nome exclusivo que identifica um site na internet, servindo como um endereço amigável para facilitar a navegação dos usuários. Ele é estruturado hierarquicamente, começando por um sufixo genérico de alto nível (como .com, .org ou .net) e seguido pelo nome específico registrado.

Na AWS, o serviço de DNS é o \textbf{Route 53}, que oferece funcionalidades avançadas, como registro de domínios, resolução de nomes DNS e verificação da integridade de serviços. O Route 53 também suporta balanceamento de carga e failover, tornando-o uma solução robusta para gerenciar tráfego em sistemas distribuídos e garantir alta disponibilidade.

\subsubsection{IaC e Terraform}
% 🆗 Revisado

% O que é IaC?
% O que é Terraform?
% Como IaC beneficia o desenvolvimento de aplicações web?

\emph{IaC (Infrastructure as Code)} é a prática de gerenciar e estruturar infraestrutura cloud por meio de arquivos de configuração, eliminando a necessidade de configuração manual. Essa abordagem automatiza a criação, modificação e destruição de recursos, tornando os processos mais eficientes e menos propensos a erros humanos\cite{what-is-iac}.

Os benefícios do IaC no desenvolvimento de aplicações web incluem a automação de processos, reduzindo a probabilidade de erros e aumentando a eficiência operacional. Ele também proporciona consistência ao permitir que ambientes sejam replicados de maneira padronizada, seja para desenvolvimento, testes ou produção. Além disso, a prática facilita a colaboração entre equipes, já que os arquivos de configuração podem ser versionados e modificados de forma controlada. 

O \emph{Terraform} é uma ferramenta de IaC que permite definir e gerenciar infraestrutura em múltiplos provedores de nuvem, como AWS, Azure e Google Cloud, usando arquivos de configuração escritos em linguagem declarativa\cite{terraform-docs}. Com ele, é possível criar ambientes completos e replicáveis, desde servidores até redes e bancos de dados, de forma consistente e escalável.

Uma das principais vantagens do uso do Terraform é a capacidade de descrever toda a arquitetura em arquivos de configuração que podem ser facilmente armazenados, revisados e auditados. Ele utiliza sua própria linguagem, HCL (HashiCorp Configuration Language), que permite definir recursos da AWS, como instâncias EC2, balanceadores de carga, grupos de segurança e outros componentes necessários. O Terraform então interage diretamente com a API da AWS para provisionar e gerenciar esses recursos.




\subsection{Segurança}
% 🆗 Revisado

A segurança é um pilar essencial no desenvolvimento e operação de aplicações web, englobando práticas e tecnologias destinadas a proteger sistemas, dados e usuários contra acessos não autorizados, ataques e outras ameaças.


\subsubsection{Criptografia}
% 🆗 Revisado

% O que é criptografia?
% O que é criptografia em repouso?
% O que é criptografia em trânsito?

Criptografia é o processo de proteger informações ao transformá-las em um formato ilegível para indivíduos não autorizados, utilizando algoritmos e chaves criptográficas para cifrar e decifrar os dados. Em aplicações web, a criptografia pode ser categorizada em dois tipos principais:

\begin{itemize}
    \item \textbf{Criptografia em repouso}: Garante a proteção de dados armazenados, como discos rígidos e bancos de dados, prevenindo o acesso não autorizado mesmo em caso de comprometimento físico dos dispositivos.
    \item \textbf{Criptografia em trânsito}: Protege dados enquanto são transmitidos entre sistemas, garantindo a confidencialidade e integridade das informações contra interceptações ou ataques \emph{man-in-the-middle}, onde um invasor monitora ou modifica a comunicação entre duas partes.
\end{itemize}

\subsubsection{Certificados SSL}
% 🆗 Revisado

% O que é um certificado SSL?
% O que é HTTPS?

Um certificado \emph{SSL (Secure Sockets Layer)} é um arquivo digital que autentica a identidade de um site e habilita conexões seguras por meio de criptografia. Sua utilização resulta no protocolo \emph{HTTPS (HyperText Transfer Protocol Secure)}, que assegura a transmissão segura de dados entre o servidor e os usuários, impedindo interceptações e garantindo confiança.

\subsubsection{SSH}
% 🆗 Revisado

% O que é SSH?

O \emph{SSH (Secure Shell)} é um protocolo de rede que permite o acesso remoto seguro a máquinas e servidores, criptografando as comunicações para proteger contra interceptações e ataques. Ele é amplamente utilizado para executar comandos remotamente, gerenciar servidores e realizar transferências de arquivos de maneira segura.

\subsubsection{Firewalls}
% 🆗 Revisado

% O que é um firewall?
% O que é um grupo de segurança na AWS? Como funciona?

Um \emph{firewall} é um sistema que regula o tráfego de rede com base em regras de segurança predefinidas, permitindo ou bloqueando comunicações conforme necessário.

Na AWS, os grupos de segurança atuam como firewalls virtuais para recursos, como instâncias EC2, controlando o tráfego de entrada e saída. Eles funcionam por meio de regras que especificam protocolos, portas e origens ou destinos permitidos, assegurando que as instâncias estejam protegidas contra acessos não autorizados e tráfego indesejado.

Essas práticas e tecnologias são indispensáveis para proteger aplicações web contra ameaças e vulnerabilidades, promovendo confiança, integridade e disponibilidade dos sistemas.









