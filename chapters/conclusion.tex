\chapter*{Conclusões}

Este trabalho analisou a escalabilidade da plataforma Codeboard UERJ, uma aplicação web colaborativa para edição de código em tempo real, projetada para um contexto global de ensino e aprendizado. O estudo teve como objetivo identificar os principais desafios enfrentados pela plataforma em termos de desempenho e disponibilidade, propondo soluções para otimizar a infraestrutura e garantir uma experiência de usuário satisfatória, mesmo em cenários de alta demanda e instabilidade.

Os testes feitos foram fundamentais para validar a resiliência da infraestrutura. Os resultados demonstraram que a plataforma é capaz de responder de forma eficiente às variações de carga, apresentando tempos de resposta adequados e capacidade de recuperação automática de falhas em processos e servidores, sem interrupções significativas para os usuários. Além disso, as soluções de segurança implementadas comprovaram a proteção eficaz contra possíveis ataques, destacando a confiabilidade do sistema.

Embora a arquitetura atual tenha se mostrado eficiente, ainda existem oportunidades de aprimoramento. Uma delas seria a adoção de máquinas virtuais mais robustas, com maior capacidade de processamento, possibilitando um melhor suporte a um número elevado de usuários e evitando sobrecargas em momentos de demanda explosiva. Essa abordagem minimizaria a necessidade de provisionamento imediato de novos servidores, otimizando o desempenho do sistema em situações críticas.

Outro avanço relevante seria a redução do tempo de inicialização de novas máquinas virtuais. Diminuir o \emph{warmup time} tornaria a infraestrutura ainda mais responsiva a picos de demanda, assegurando a disponibilidade dos recursos com maior agilidade. Implementar essas melhorias fortaleceria a resiliência e a escalabilidade da plataforma, elevando sua eficiência e capacidade operacional a novos níveis.

Com essas evoluções, a plataforma não apenas atenderá a um público maior, mas também poderá servir como referência para o desenvolvimento de outras aplicações colaborativas em tempo real. Isso contribuirá para o avanço da computação distribuída e da educação online. A escalabilidade é um desafio constante, mas, com a abordagem correta e o uso de tecnologias modernas, é possível superá-lo e alcançar novos patamares nesta era digital.