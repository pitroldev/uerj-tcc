\chapter*{Introdução}

Na era digital em que vivemos, as aplicações em tempo real tornaram-se um componente crucial que conecta a sociedade moderna. Seja em redes sociais, jogos online ou plataformas de streaming, a busca por experiências interativas e instantâneas é constante e, durante a pandemia de COVID-19, a necessidade dessas plataformas digitais tornou-se ainda mais evidente, especialmente no campo educacional \cite{impact-covid19-teaching-learning}. Com o fechamento das instituições de ensino e a transição para o formato remoto, tecnologias capazes de oferecer colaboração em tempo real ganharam destaque como elementos essenciais para manter a continuidade do aprendizado.

A escalabilidade, nesse cenário, se mostra como um dos maiores desafios técnicos para o desenvolvimento de sistemas globais, pois é necessário atender a um número crescente de usuários e à crescente complexidade das interações em tempo real. Soluções robustas em computação em nuvem e estratégias de paralelismo têm se mostrado indispensáveis para superar esses obstáculos e garantir desempenho e confiabilidade mesmo em condições de alta demanda.

Visando contribuir para a superação desses desafios e fornecer uma experiência de ensino remoto que empodere professores e alunos, este projeto de graduação tem como objetivo explorar as nuances da escalabilidade em aplicações globais em tempo real, com foco no desenvolvimento da plataforma Codeboard UERJ. A plataforma oferece um ambiente de programação colaborativa em tempo real, projetado para apoiar o ensino em aulas remotas, possibilitando a interação simultânea entre professores e alunos. Ao longo do trabalho, serão investigadas práticas e estratégias que garantam uma infraestrutura capaz de se adaptar à demanda, mantendo um alto nível de desempenho e confiabilidade em cenários de uso intensivo ou em casos de falhas inesperadas.

Ao apresentar os desafios enfrentados e as soluções implementadas, este estudo busca contribuir para a compreensão e o avanço da engenharia de sistemas em larga escala. Além disso, espera-se que os insights obtidos sirvam como referência prática para desenvolvedores, arquitetos de sistemas e tomadores de decisão, promovendo a excelência na entrega de serviços em tempo real e fortalecendo a conexão entre tecnologia e educação.

\section*{Estrutura do Trabalho}

Este trabalho está organizado em cinco capítulos, conforme descrito a seguir:

O Capítulo 1 apresenta os conceitos básicos relacionados ao desenvolvimento web, protocolos de comunicação, bancos de dados, computação em nuvem e segurança. Estes conceitos são fundamentais para a compreensão do desenvolvimento e da infraestrutura da plataforma Codeboard UERJ, bem como para a análise dos testes realizados.

No Capítulo 2, é apresentada a plataforma Codeboard UERJ, com detalhes sobre seus objetivos, funcionalidades e implementação em termos de back-end, front-end, banco de dados e comunicação em tempo real.

O Capítulo 3 aborda a infraestrutura cloud implementada na plataforma, entrando em detalhes sobre o planejamento, a implementação e a segurança da arquitetura utilizada. São discutidos aspectos como infraestrutura de computação, balanceamento de carga, escalabilidade, recuperação de falhas, banco de dados, segurança e monitoramento.

O Capítulo 4 descreve os testes realizados na plataforma, incluindo testes de escalabilidade, tolerância a falhas, desempenho, segurança e monitoramento. Neste capítulo, são apresentados os resultados obtidos e conclusões sobre a eficácia das soluções implementadas.

Por fim, a conclusão do trabalho apresenta uma síntese dos resultados obtidos, discute as contribuições do estudo e sugere possíveis direções para trabalhos futuros.

