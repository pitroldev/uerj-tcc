\section*{Introdução}

Na era digital em que vivemos, as aplicações em tempo real tornaram-se um componente crucial que conecta a sociedade moderna. Durante a pandemia de COVID-19, a necessidade de ferramentas digitais interativas e de qualidade tornou-se ainda mais evidente, especialmente no campo educacional\cite{impact-covid19-teaching-learning}. Com o fechamento das instituições de ensino e a transição para o formato remoto, tecnologias capazes de oferecer colaboração em tempo real ganharam destaque como elementos essenciais para manter a continuidade do aprendizado.

A escalabilidade, nesse cenário, se mostra como um dos maiores desafios técnicos para o desenvolvimento de sistemas globais, pois é necessário atender a um número crescente de usuários e à crescente complexidade das interações em tempo real. Soluções robustas em computação em nuvem e estratégias de paralelismo têm se mostrado indispensáveis para superar esses obstáculos e garantir desempenho e confiabilidade mesmo em condições de alta demanda.

Este projeto de graduação tem como objetivo explorar as nuances da escalabilidade em aplicações globais em tempo real, com foco na análise e desenvolvimento da plataforma Codeboard UERJ. Voltada para o ensino de programação, a plataforma permite a colaboração simultânea de professores e alunos por meio de um ambiente de codificação interativo e dinâmico. Ao longo do trabalho, serão investigadas práticas e estratégias que garantem uma infraestrutura capaz de se adaptar à demanda, mantendo um alto nível de desempenho e confiabilidade.

Ao documentar os desafios enfrentados e as soluções implementadas, este estudo busca contribuir para a compreensão e o avanço da engenharia de sistemas em larga escala. Além disso, espera-se que os insights obtidos sirvam como referência prática para desenvolvedores, arquitetos de sistemas e tomadores de decisão, promovendo a excelência na entrega de serviços em tempo real e fortalecendo a conexão entre tecnologia e educação.

Nos capítulos que seguem, serão apresentados os fundamentos teóricos que embasam este trabalho, incluindo conceitos de desenvolvimento web, computação em nuvem e protocolos de comunicação. Em seguida, será detalhada a arquitetura e implementação da plataforma Codeboard UERJ, abordando as estratégias adotadas para garantir sua robustez e eficiência. Os capítulos posteriores trarão uma análise dos testes realizados, os resultados obtidos e as discussões sobre as soluções implementadas. Por fim, serão destacadas as conclusões alcançadas e as perspectivas para trabalhos futuros.